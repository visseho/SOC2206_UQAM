% Options for packages loaded elsewhere
\PassOptionsToPackage{unicode}{hyperref}
\PassOptionsToPackage{hyphens}{url}
%
\documentclass[
]{article}
\usepackage{lmodern}
\usepackage{amssymb,amsmath}
\usepackage{ifxetex,ifluatex}
\ifnum 0\ifxetex 1\fi\ifluatex 1\fi=0 % if pdftex
  \usepackage[T1]{fontenc}
  \usepackage[utf8]{inputenc}
  \usepackage{textcomp} % provide euro and other symbols
\else % if luatex or xetex
  \usepackage{unicode-math}
  \defaultfontfeatures{Scale=MatchLowercase}
  \defaultfontfeatures[\rmfamily]{Ligatures=TeX,Scale=1}
\fi
% Use upquote if available, for straight quotes in verbatim environments
\IfFileExists{upquote.sty}{\usepackage{upquote}}{}
\IfFileExists{microtype.sty}{% use microtype if available
  \usepackage[]{microtype}
  \UseMicrotypeSet[protrusion]{basicmath} % disable protrusion for tt fonts
}{}
\makeatletter
\@ifundefined{KOMAClassName}{% if non-KOMA class
  \IfFileExists{parskip.sty}{%
    \usepackage{parskip}
  }{% else
    \setlength{\parindent}{0pt}
    \setlength{\parskip}{6pt plus 2pt minus 1pt}}
}{% if KOMA class
  \KOMAoptions{parskip=half}}
\makeatother
\usepackage{xcolor}
\IfFileExists{xurl.sty}{\usepackage{xurl}}{} % add URL line breaks if available
\IfFileExists{bookmark.sty}{\usepackage{bookmark}}{\usepackage{hyperref}}
\hypersetup{
  pdftitle={Labo 4: Paramètres de tendance centrale},
  pdfauthor={Visseho Adjiwanou, PhD.},
  hidelinks,
  pdfcreator={LaTeX via pandoc}}
\urlstyle{same} % disable monospaced font for URLs
\usepackage[margin=1in]{geometry}
\usepackage{longtable,booktabs}
% Correct order of tables after \paragraph or \subparagraph
\usepackage{etoolbox}
\makeatletter
\patchcmd\longtable{\par}{\if@noskipsec\mbox{}\fi\par}{}{}
\makeatother
% Allow footnotes in longtable head/foot
\IfFileExists{footnotehyper.sty}{\usepackage{footnotehyper}}{\usepackage{footnote}}
\makesavenoteenv{longtable}
\usepackage{graphicx,grffile}
\makeatletter
\def\maxwidth{\ifdim\Gin@nat@width>\linewidth\linewidth\else\Gin@nat@width\fi}
\def\maxheight{\ifdim\Gin@nat@height>\textheight\textheight\else\Gin@nat@height\fi}
\makeatother
% Scale images if necessary, so that they will not overflow the page
% margins by default, and it is still possible to overwrite the defaults
% using explicit options in \includegraphics[width, height, ...]{}
\setkeys{Gin}{width=\maxwidth,height=\maxheight,keepaspectratio}
% Set default figure placement to htbp
\makeatletter
\def\fps@figure{htbp}
\makeatother
\setlength{\emergencystretch}{3em} % prevent overfull lines
\providecommand{\tightlist}{%
  \setlength{\itemsep}{0pt}\setlength{\parskip}{0pt}}
\setcounter{secnumdepth}{-\maxdimen} % remove section numbering

\title{Labo 4: Paramètres de tendance centrale}
\author{Visseho Adjiwanou, PhD.}
\date{01 February 2022}

\begin{document}
\maketitle

\hypertarget{partie-a}{%
\section{PARTIE A}\label{partie-a}}

Voici les résultats obtenus au cours d'une enquête sur l'âge et le
statut matrimoniale des répondants.

\textbf{Tableau 1: Distribution du statut matrimonial}

\begin{longtable}[]{@{}lllll@{}}
\toprule
Statut & Fréquence & Pourcentage & Pourcentage valide & Pourcentage
cumulé\tabularnewline
\midrule
\endhead
Marié & 247 & 29.1 & 29.1 & 29.1\tabularnewline
Veuf & 3 & .4 & .4 & 29.4\tabularnewline
Divorcé & 36 & 4.2 & 4.2 & 33.6\tabularnewline
Séparé & 14 & 1.6 & 1.6 & 35.3\tabularnewline
Jamais marié & 550 & 64.7 & 64.7 & 100.0\tabularnewline
\textbf{Total} & \textbf{850} & \textbf{100.0} & \textbf{100.0}
&\tabularnewline
\bottomrule
\end{longtable}

Répondez aux questions suivantes:

\begin{enumerate}
\def\labelenumi{\arabic{enumi}.}
\tightlist
\item
  Quel est le type de la variable étudiée?
\item
  Quel est la valeur du mode
\item
  Si vous pouvez utiliser la médiane, indiquez sa valeur. Sinon, dites
  que ce n'est pas possible et expliquer votre réponse.
\item
  Si vous pouvez utiliser la moyenne, indiquez sa valeur. Sinon, dites
  que ce n'est pas possible et expliquer votre réponse.
\item
  Quel est le problème avec ce tableau? Quelle solution préconisez-vous?
\end{enumerate}

Au cours de la même enquête, on a collecté les données sur le groupe
d'âges des enquêtés. les résultats sont présentés dans le tableau 2.

\textbf{Tableau 2: Distribution du groupe d'âges}

\begin{longtable}[]{@{}lllll@{}}
\toprule
Groupe d'âge & Fréquence & Pourcentage & Pourcentage valide &
Pourcentage cumulé\tabularnewline
\midrule
\endhead
15 - 24 & 276 & & &\tabularnewline
25 - 34 & 199 & & &\tabularnewline
35 - 49 & 263 & & &\tabularnewline
50 et plus & 77 & & &\tabularnewline
Non réponse & 35 & & &\tabularnewline
\textbf{Total} & \textbf{850} & & &\tabularnewline
\bottomrule
\end{longtable}

\begin{enumerate}
\def\labelenumi{\arabic{enumi}.}
\tightlist
\item
  Quel est le type de la variable étudiée?
\item
  Complétez le tableau
\item
  Quel est la valeur du mode?
\item
  Si vous pouvez utiliser la médiane, indiquez sa valeur. Sinon, dites
  que ce n'est pas possible et expliquer votre réponse.
\item
  Si vous pouvez utiliser la moyenne, indiquez sa valeur. Sinon, dites
  que ce n'est pas possible et expliquer votre réponse.
\end{enumerate}

\hypertarget{question-2}{%
\subsection{Question 2}\label{question-2}}

Voici les données issues d'une enquête dans une classe

\begin{longtable}[]{@{}ll@{}}
\toprule
Age & Nombre d'élèves\tabularnewline
\midrule
\endhead
10 & 5\tabularnewline
11 & 7\tabularnewline
12 & 4\tabularnewline
\bottomrule
\end{longtable}

\begin{enumerate}
\def\labelenumi{\arabic{enumi}.}
\tightlist
\item
  Quelle est la variable étudiée?
\item
  Quelle est la valeur de l'âge moyen de la classe?
\item
  Quelle est la valeur de l'âge modal de la classe?
\item
  Quelle est la valeur de l'âge médian de la classe?
\end{enumerate}

\hypertarget{partie-b}{%
\section{PARTIE B}\label{partie-b}}

\hypertarget{la-solution-technologique-au-changement-climatique-exemple-tiruxe9-de-krieg}{%
\section{La solution technologique au changement climatique (exemple
tiré de
Krieg)}\label{la-solution-technologique-au-changement-climatique-exemple-tiruxe9-de-krieg}}

Beaucoup de gens pensent qu'en adoptant de nouvelles technologies, nous
pouvons économiser à la fois de l'argent et protéger l'environnement en
brûlant moins de combustibles fossiles. Cet exercice est tiré du livre
de krieg, ``Statistics and data analysis for Social Science''.

\hypertarget{que-pensez-vous-de-cette-assertion}{%
\subsection{1. Que pensez-vous de cette
assertion?}\label{que-pensez-vous-de-cette-assertion}}

\hypertarget{en-quoi-nest-elle-pas-valide}{%
\subsection{2. En quoi n'est-elle pas
valide?}\label{en-quoi-nest-elle-pas-valide}}

Pour tester cette assertion, nous utilisons les données de 1994 et de
2009 sur les voitures les plus efficients entre les deux périodes. Le
tableau suivant présente les vitesses (mile per gallon, mpg) pour les
différentes marques de voitures pour leur circulation en ville et sur
l'autoroute:

\begin{itemize}
\tightlist
\item
  \textbf{Pour 1994}
\end{itemize}

\begin{longtable}[]{@{}lll@{}}
\toprule
Marque et modèle & Ville (mpg) & Autoroute(mpg)\tabularnewline
\midrule
\endhead
Mazda 626 & 23 & 31\tabularnewline
Honda Accord & 22 & 29\tabularnewline
Chevrolet Corsica & 22 & 28\tabularnewline
Buick Century & 22 & 28\tabularnewline
Oldsmobile Cutlass Ciera & 22 & 28\tabularnewline
Oldsmobile Achieva & 21 & 32\tabularnewline
Pontiac Grand Am & 21 & 32\tabularnewline
Infiniti G20 & 21 & 29\tabularnewline
Mitsubishi Galant & 21 & 28\tabularnewline
Dodge Spirit & 21 & 27\tabularnewline
Plymouth Acclaim & 21 & 27\tabularnewline
Subaru Legacy & 20 & 28\tabularnewline
Toyota Camry & 20 & 27\tabularnewline
Hyundai Sonata & 19 & 26\tabularnewline
Chrysler LeBaron & 19 & 25\tabularnewline
Ford Taurus & 18 & 27\tabularnewline
Mercury Sable & 18 & 27\tabularnewline
Eagle Vision & 18 & 26\tabularnewline
\bottomrule
\end{longtable}

\begin{itemize}
\tightlist
\item
  \textbf{Pour 2009}
\end{itemize}

\begin{longtable}[]{@{}lll@{}}
\toprule
Marque et modèle & Ville (mpg) & Autoroute(mpg)\tabularnewline
\midrule
\endhead
Toyota Prius Hybrid) & 48 & 45\tabularnewline
Nissan Altima (hybrid) & 35 & 33\tabularnewline
Toyota Camry (hybrid) & 33 & 34\tabularnewline
Chevrolet Malibu (hybrid) & 26 & 34\tabularnewline
Saturn Aura (hybrid) & 26 & 34\tabularnewline
Hyundai Elantra & 25 & 33\tabularnewline
Kia Spectra & 24 & 32\tabularnewline
Nissan Altima & 23 & 32\tabularnewline
Saturn Aura & 22 & 33\tabularnewline
Kia Optima & 22 & -\tabularnewline
Hyundai Sonata & 22 & 32\tabularnewline
Honda Accord & 22 & 31\tabularnewline
Chevrolet Malibu & 22 & 30\tabularnewline
Toyota Camry & 21 & 31\tabularnewline
Volkswagen Passat & 21 & 31\tabularnewline
Mazda 6 & 21 & 30\tabularnewline
Chrysler Sebring & 21 & 30\tabularnewline
Dodge Avenger & 21 & 30\tabularnewline
Ford Fusion & 20 & 29\tabularnewline
Mercury Milan & 20 & 29\tabularnewline
Mitsubishi Galant & 20 & 27\tabularnewline
Subaru Legacy & 20 & 27\tabularnewline
Nissan Maxima & 19 & 260\tabularnewline
Nissan Altima & 19 & 26\tabularnewline
Mercury Sable & 18 & 28\tabularnewline
Hyundai Azera & 18 & 26\tabularnewline
Buick LaCrosse/Allure & 17 & 28\tabularnewline
\bottomrule
\end{longtable}

\hypertarget{quelle-est-la-taille-de-chaque-uxe9chantillon}{%
\subsection{3. Quelle est la taille de chaque
échantillon}\label{quelle-est-la-taille-de-chaque-uxe9chantillon}}

\hypertarget{uxe9fficacituxe9-gagnuxe9e-en-ville}{%
\subsection{4. Éfficacité gagnée en
ville}\label{uxe9fficacituxe9-gagnuxe9e-en-ville}}

Vous allez calculé le mode, la médiane et la moyenne pour la vitesse en
\textbf{ville} en 1994 et 2009. Quelle conclusion tirez-vous? A cette
étape de l'exercice, je vous demande de faire les calculs à la main.

\hypertarget{uxe9fficacituxe9-sur-autoroute}{%
\subsection{5. Éfficacité sur
autoroute}\label{uxe9fficacituxe9-sur-autoroute}}

Le calcul que vous venez de faire est trop long. On peut présenter les
données précédentes sous forme de données agrégées. C'est quoi encore
les données agrégées?

4.1 Regrouper les données de la \textbf{vitesse sur l'autoroute} sous
forme agrégée. Cela veut dire qu'il faut dénombrer le nombre de voitures
pour chaque niveau de vitesse. Faite cela pour les données de 1994 et de
2009.

\hypertarget{pruxe9senter-dans-ce-muxeame-tableau-les-fruxe9quences-et-les-fruxe9quences-cumuluxe9es}{%
\subsubsection{5.1 Présenter dans ce même tableau les fréquences, et les
fréquences
cumulées}\label{pruxe9senter-dans-ce-muxeame-tableau-les-fruxe9quences-et-les-fruxe9quences-cumuluxe9es}}

\hypertarget{quelle-repruxe9sentation-graphique-vous-semble-la-plus-appropriuxe9e-pour-ces-donnuxe9es}{%
\subsubsection{5.2 Quelle représentation graphique vous semble la plus
appropriée pour ces
données?}\label{quelle-repruxe9sentation-graphique-vous-semble-la-plus-appropriuxe9e-pour-ces-donnuxe9es}}

\hypertarget{calculer-uxe0-nouveau-le-mode-la-muxe9diane-et-la-moyenne-uxe0-partir-de-ses-donnuxe9es-groupuxe9es.-quelle-conclusion-tirez-vous}{%
\subsubsection{5.3 Calculer à nouveau le mode, la médiane et la moyenne
à partir de ses données groupées. Quelle conclusion
tirez-vous?}\label{calculer-uxe0-nouveau-le-mode-la-muxe9diane-et-la-moyenne-uxe0-partir-de-ses-donnuxe9es-groupuxe9es.-quelle-conclusion-tirez-vous}}

--------------- PAS NÉCESSAIRE LA QUESTION 6-------------------

\hypertarget{utilisation-de-r}{%
\subsection{6. Utilisation de R}\label{utilisation-de-r}}

Maintenant, nous allons utiliser R pour faire le même travail. Voici
comment vous allez procéder.

\begin{enumerate}
\def\labelenumi{\arabic{enumi}.}
\tightlist
\item
  Créer la base de données \textbf{donnee\_1994} avec les variable
  suivantes:
\end{enumerate}

\begin{itemize}
\tightlist
\item
  modele
\item
  vitesse\_ville et
\item
  vitesse\_autoroute
\end{itemize}

Vous comprenez que cette base de données contient donc 18 observations
pour 3 variables. Quelle est la nature de chaque variable?

\hypertarget{ruxe9ponse-1}{%
\subsubsection{Réponse 1}\label{ruxe9ponse-1}}

\begin{enumerate}
\def\labelenumi{\arabic{enumi}.}
\setcounter{enumi}{1}
\tightlist
\item
  Calculer la moyenne, la médiane et le mode des deux variables
  \textbf{vitesse\_ville} et \textbf{vitesse\_autoroute} à partir des
  données \textbf{donnee\_1994}.
\end{enumerate}

\begin{itemize}
\tightlist
\item
  Commenter vos résultats. Si vous vous rappelez, pour calculer la
  moyenne et la médiane, il faut utiliser les fonctions \textbf{mean} et
  \textbf{mediane}.
\item
  Cependant, il \textbf{N'existe PAS} de fonction \textbf{mode} pour
  calculer le mode. Je vous demande de faire quelques recherches et me
  venir avec une solution. Il est dès fois important de ne pas se
  focaliser pour comprendre ce que vous faites du moment où ça marche.
  Donnez-vous le temps de le comprendre plus tard.
\end{itemize}

\begin{enumerate}
\def\labelenumi{\arabic{enumi}.}
\setcounter{enumi}{2}
\item
  Il y a plusieurs autres paramètres de tendance centrale que les trois
  que nous avons vus en classe. Vous avez le minimum, le maximum, le
  premier quartile, le 3e quartile et plus généralement les
  \textbf{ntiles}. Calculer ces différents paramètres sur les variables
  vitesse\_ville et vitesse\_autoroute. Commenter vos résultats.
\item
  La fonction \textbf{descr} de summarytools vous permet aussi de
  calculer ces paramètres de tendance centrale. Utiliser cette fonction
  pour calculer les paramètres calculer au 2 et 3.
\item
  Maintenant, refaite la même chose avec les données de 2009.
\item
  Quelle conclusion tirez-vous sur la solution technologique au
  changement climatique?
\end{enumerate}

\end{document}
