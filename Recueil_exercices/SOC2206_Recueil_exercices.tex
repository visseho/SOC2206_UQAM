% Options for packages loaded elsewhere
\PassOptionsToPackage{unicode}{hyperref}
\PassOptionsToPackage{hyphens}{url}
%
\documentclass[
]{article}
\usepackage{amsmath,amssymb}
\usepackage{lmodern}
\usepackage{iftex}
\ifPDFTeX
  \usepackage[T1]{fontenc}
  \usepackage[utf8]{inputenc}
  \usepackage{textcomp} % provide euro and other symbols
\else % if luatex or xetex
  \usepackage{unicode-math}
  \defaultfontfeatures{Scale=MatchLowercase}
  \defaultfontfeatures[\rmfamily]{Ligatures=TeX,Scale=1}
\fi
% Use upquote if available, for straight quotes in verbatim environments
\IfFileExists{upquote.sty}{\usepackage{upquote}}{}
\IfFileExists{microtype.sty}{% use microtype if available
  \usepackage[]{microtype}
  \UseMicrotypeSet[protrusion]{basicmath} % disable protrusion for tt fonts
}{}
\makeatletter
\@ifundefined{KOMAClassName}{% if non-KOMA class
  \IfFileExists{parskip.sty}{%
    \usepackage{parskip}
  }{% else
    \setlength{\parindent}{0pt}
    \setlength{\parskip}{6pt plus 2pt minus 1pt}}
}{% if KOMA class
  \KOMAoptions{parskip=half}}
\makeatother
\usepackage{xcolor}
\usepackage[margin=1in]{geometry}
\usepackage{graphicx}
\makeatletter
\def\maxwidth{\ifdim\Gin@nat@width>\linewidth\linewidth\else\Gin@nat@width\fi}
\def\maxheight{\ifdim\Gin@nat@height>\textheight\textheight\else\Gin@nat@height\fi}
\makeatother
% Scale images if necessary, so that they will not overflow the page
% margins by default, and it is still possible to overwrite the defaults
% using explicit options in \includegraphics[width, height, ...]{}
\setkeys{Gin}{width=\maxwidth,height=\maxheight,keepaspectratio}
% Set default figure placement to htbp
\makeatletter
\def\fps@figure{htbp}
\makeatother
\setlength{\emergencystretch}{3em} % prevent overfull lines
\providecommand{\tightlist}{%
  \setlength{\itemsep}{0pt}\setlength{\parskip}{0pt}}
\setcounter{secnumdepth}{-\maxdimen} % remove section numbering
\ifLuaTeX
  \usepackage{selnolig}  % disable illegal ligatures
\fi
\IfFileExists{bookmark.sty}{\usepackage{bookmark}}{\usepackage{hyperref}}
\IfFileExists{xurl.sty}{\usepackage{xurl}}{} % add URL line breaks if available
\urlstyle{same} % disable monospaced font for URLs
\hypersetup{
  pdfauthor={Visseho Adjiwanou; Sociology department, Université du Québec à Montréal (UQAM); Aoudou Njingouo Mounchingam (Contribution); Yao Jean Kouadio (Contribution)},
  hidelinks,
  pdfcreator={LaTeX via pandoc}}

\title{\vspace{3.5in}

``Introduction à la statistique sociale?''}
\usepackage{etoolbox}
\makeatletter
\providecommand{\subtitle}[1]{% add subtitle to \maketitle
  \apptocmd{\@title}{\par {\large #1 \par}}{}{}
}
\makeatother
\subtitle{Recueil d'exercices}
\author{Visseho Adjiwanou\footnote{corresponding author -
  \href{mailto:adjiwanou.vissého@uqam.ca}{\nolinkurl{adjiwanou.vissého@uqam.ca}}} \and Sociology
department, Université du Québec à Montréal (UQAM) \and Aoudou Njingouo
Mounchingam (Contribution) \and Yao Jean Kouadio (Contribution)}
\date{18 November 2022}

\begin{document}
\maketitle

\newpage 
\tableofcontents 
\listoffigures
\listoftables
\newpage

\hypertarget{table-des-matiuxe8res}{%
\section{Table des matières}\label{table-des-matiuxe8res}}

\hypertarget{chapitre-1-statistiques-et-variables}{%
\section{Chapitre 1: Statistiques et
variables}\label{chapitre-1-statistiques-et-variables}}

\hypertarget{exercice-1-du-concept-uxe0-la-mesure}{%
\subsection{Exercice 1: Du concept à la
mesure}\label{exercice-1-du-concept-uxe0-la-mesure}}

Marx a soutenu que le capitalisme est un mode de production économique
basé sur le conflit entre deux classes de personnes: le capital (alias
la bourgeoisie) et le travail (alias le prolétariat). L'appartenance à
une classe est déterminée par sa relation avec les moyens de production.
Le capital possède les moyens de production et le travail non. Au
travail, les membres de la classe ouvrière vendent leur capacité de
faire du travail (leur force de travail) au capital en échange d'un taux
de salaire. Marx a affirmé que cela a pour effet d'aliéner les gens de
leur travail parce que les individus ne contrôlent plus leur travail, ni
le produit de leur travail.

Logiquement, cela a du sens; cependant, pouvons-nous mesurer
l'aliénation d'une manière ou d'une autre pour offrir des preuves
empiriques de son existence? L'aliénation de Marx existe au niveau
conceptuel. Pouvez-vous penser à des moyens d'opérationnaliser
l'aliénation?

\end{document}
